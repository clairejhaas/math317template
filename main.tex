% Left align equations and labels equations on the left
\documentclass[fleqn, leqno]{article}
\usepackage[margin=1in]{geometry}
\usepackage[utf8]{inputenc}
\usepackage{amsmath}
\usepackage{amssymb}
\usepackage{amsthm}
\usepackage{fancyhdr}
\usepackage{chngcntr}
% Creates header with name, course, assignment, and page number. Removes footer page number
\pagestyle{fancy}
\fancyhead[L]{Name}
\fancyhead[C]{MATH 317: Homework 1}
\fancyhead[R]{\thepage}
\cfoot{}
% Changes the subsection to be alphabetical characters as seen in textbook
\renewcommand*{\thesubsection}{\alph{subsection}.}
% Changes the equation numbering to restart at section
\counterwithin*{equation}{section}
% Defines the norm symbol
\newcommand{\norm}[1]{\lVert#1\rVert}
% Defines the fancy R symbol
\newcommand{\R}{\mathbb{R}}
\newcommand{\realnumberset}{\mathbb{R}^n}

\begin{document}
\section*{This is a non-numbered section}
This is a matrix $x = \begin{bmatrix}
    -2 & 4 & 5 \\
    0 & 0 & 0
\end{bmatrix}
$
\subsection{This is a subsection}
\subsection{And another subsection}
\subsection{Let's align by the equal signs, use the \&= command}
\begin{align*}
0 &= \begin{bmatrix}
0 & 1 & 2
\end{bmatrix}\\
0 &= \begin{bmatrix}
    0 & 1 & 2
    \end{bmatrix}\\
0 + 0 &= \begin{bmatrix}
    0 & 1 & 2
\end{bmatrix}
\end{align*}
\section{Create a numbered section}
\subsection{Enter math mode using the \$ symbol for fancy symbols}
\subsection{Some mathematical notations}
Use the carrot to create an exponent: $x^n$\\
Use the underscore to create a subscript: 
$x_n$\\
Use \verb|\norm| to create the normal symbol: 
$\norm{x}$\\
Use \verb|\qed| to create the qed square: 
$\qed$\\
Use \verb|\cdot| to create the dot product symbol: 
$\cdot$\\
Use \verb|\R| to create Fancy R: $\R$ \\
Use \verb|\realnumberset| to create real number set: $\realnumberset$ \\
Use \verb|\vec| to create a vector: 
$\Vec{x}$\\
Use \verb|\times| to create a cross product: 
$\times$
\subsection{Some text formatting}
Use \verb|\textbf| to create \textbf{some bold text}\\
Use \verb|\textit| to create \textit{some italics}\\
Use \verb|\text| to create normal text inside of math mode
\subsection{Use $\backslash \backslash$ to create a new line}
\end{document}
